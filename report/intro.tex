\section{Introduction}
\label{section:introduction}

Today WPA-2 Enterpise stack of protocols is defacto standard 
for building secure WiFi networks. For that matter in this 
document we describe how to build commercial product using 
these standards. Our primary goal, however, is hands on experience.

In this pace, this document has the following structure.
First, we give definitions to identification, authentication, authorization,
and accounting. Next, we discuss the extensible authentication framework (EAP).
We then turn our attention to RADIUS protocol and discuss interaction 
of EAP-TTLS protocol. Next, we switch to discussion of wireless encryption protocols used in
modern WiFi networks. We then move on to discussion of overall 
architecture of the system. For that matter we discuss the interactions with RADIUS server,
how billing is organized and finally scratch the surface regarding
monetization issues. We then describe what kind of hardware and software 
we have used to build the prototype solution. 
Next, to get an idea of how the system will perform in real-life settings 
we sat down and measured the performance of basic 
operations such as encryption, decryption and key derivation. Here we will 
report such metrics as duration of protocol execution and energy 
consumed during the protocol run. We conclude the document with the 
discussion of how the system can be deployed in public places. 
For example, here we touch such aspects as voucher production and 
distribution.