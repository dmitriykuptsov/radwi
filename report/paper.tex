\documentclass[conference,10pt,letter]{IEEEtran}

\usepackage{url}
\usepackage{amssymb,amsthm}
\usepackage{graphicx,color}

%\usepackage{balance}

\usepackage{cite}
\usepackage{amsmath}
\usepackage{amssymb}

\usepackage{color, colortbl}
\usepackage{times}
\usepackage{caption}
\usepackage{rotating}
\usepackage{subcaption}

\newtheorem{theorem}{Theorem}
\newtheorem{example}{Example}
\newtheorem{definition}{Definition}
\newtheorem{lemma}{Lemma}

\newcommand{\XXXnote}[1]{{\bf\color{red} XXX: #1}}
\newcommand{\YYYnote}[1]{{\bf\color{red} YYY: #1}}
\newcommand*{\etal}{{\it et al.}}

\newcommand{\eat}[1]{}
\newcommand{\bi}{\begin{itemize}}
\newcommand{\ei}{\end{itemize}}
\newcommand{\im}{\item}
\newcommand{\eg}{{\it e.g.}\xspace}
\newcommand{\ie}{{\it i.e.}\xspace}
\newcommand{\etc}{{\it etc.}\xspace}

\def\P{\mathop{\mathsf{P}}}
\def\E{\mathop{\mathsf{E}}}

\begin{document}
\sloppy
\title{RadWi: Experimenting with WPA-2 enterprise and RADIUS}
\maketitle
\begin{abstract}
In this document we describe how to build a simple billing platform
for WiFi using open-source tools. In our architecture we rely heavily on 
WPA-2 Enterprise stack of protocols and use our custom python-based implementation of RADIUS 
for authentication, authorization, and accounting purposes. To implement the billing part, 
we created custom software. The billing works as follows: 
(i) users receive or purchase vouchers which specify the amount of traffic 
included, and, of course, username and password; 
(ii) prior to distribution of vouchers the operator generates previously 
mentioned information (using custom tool) and stores it in the database; 
(iii) the users can connect to any Radwi 
WiFi network at any time and can freely use the Internet as long as 
balance (amount of traffic unused) is positive; and finally, (iv) 
at certain time quanta the Radius server updates the database and reduces the 
balance of the client by amount of traffic which was used. We foresee that this 
business model is viable and can be used in many public places. Although
the system can be monetized, our primary goal is experimentation and hands 
on experience in building secure Enterprise WiFi networks.
\end{abstract}
\input intro.tex
\input background.tex
\input architecture.tex
\input hardware.tex
\input experimental.tex
\input conclusions.tex
%\balance
\bibliographystyle{abbrv}
\bibliography{mybib}

\end{document}

